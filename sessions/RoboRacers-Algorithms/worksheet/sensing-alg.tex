\section*{Sensing}
Did the robot finish exactly where it started when you drove in a square? The robot has no way to "see" where it is. Just like you trying to walk in a big square whilst blindfolded. To let our robots "see", we use \emph{sensors} on them. In this part, we'll use the reflectance sensor. It shines some light onto the floor beneath it and measures how much is reflected. Let's look at the diagram below with a black line and a white background...

\begin{figure}[h]
    \centering
    \includegraphics[width=0.6\textwidth]{images/reflectance_sensor_cases.png}
    \vspace{1em}
\end{figure}

In case \textbf{A}, the sensor is shining onto white and the reflectance value will be high (close to 100). However, in case \textbf{B}, the sensor is shining onto black and the reflectance value will be low (close to 0).

\thinkbox{Think about what the sensor value might be in case \textbf{C} where the sensor is half on the black line and half on the white background.}

\begin{enumerate}
\item Open the script \vb{threshold-test.py}. You will see in the program we have added the sensor with the line  
\begin{codeblock}
line_sensor = ColorSensor(Port.S3)
\end{codeblock}
and that we measure the reflectance and show it on the robot's screen with
\begin{codeblock}
ev3.screen.print(line_sensor.reflection())
\end{codeblock}

\item Run the program and look at the reflectance values shown on the robot's screen. 
\item Move the robot (with your hand) so the sensor is directly above the line and make a note of the reflectance value. Repeat to get a reflectance of the background surface. 
\item Now open \vb{sensing.py} and input the values you measured for \code{LINE\_REF} and \code{BACKGROUND\_REF}. 
\end{enumerate}

Now the robot can \emph{sense} what is happening in the world, it needs to know how to react to what it \emph{senses}. We put together a set of rules which makes a process we call an \emph{algorithm}.

\begin{enumerate}
    \setcounter{enumi}{4} 
    \item Design an algorithm so that your robot will drive forwards and stop when it reaches the line. Write your algorithm in the program \vb{sensing.py} and test it.
\end{enumerate}