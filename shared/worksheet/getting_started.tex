\section*{Getting Started}
\subsection*{The Robot}
If the robot has a green light on top then it is powered on and waiting for you! If you do not see a green light, ask your session leader for help. The \vb{brick} in the picture below is the brains of the robot - it tells every other part of the robot what to do.

\begin{figure}[h]
    \centering
    \includegraphics[width=0.45\textwidth]{images/ev3brick_labelled.png}
\end{figure}

The \vb{up}, \vb{down}, \vb{left}, and \vb{right} buttons are not used to drive the robot. You will use them to select the robot's \vb{programs} from the screen. Let's run a program to test the robot!

\subsection*{Running a program}
Look at the small screen on the robot. It should look like the pictures below. 

\begin{enumerate}
    \item Use \vb{up}, \vb{down} and \vb{enter} on the robot to open \vb{File Browser}, then \vb{RoboChallenge}. 
    \item Find the \vb{starting.py} file and make sure it is highlighted in black.
    \item Run the program by pressing the middle \vb{enter} button. Listen closely! 
\end{enumerate}

\begin{figure}[h]
    \centering
    \includegraphics[width=0.9\textwidth]{images/run_program_manual.png}
\end{figure}

What did the robot say? It should have said ``Hello World!". This is a simple program to get you started. If the robot didn't speak or make any noise, ask your session leader for help.

\warningbox{To stop running a program, press the \vb{back} button at the bottom left of the robot screen.}

\newpage
\subsection*{Editing a program}
Programs don't write themselves. It's great that the robot said ``Hello World", but that's not your name! So next, we'll change the program to greet you personally. You will write your code on the laptop using Visual Studio Code. This IDE (Integrated Development Environment) is used by many students and professionals to write code.

Look at the laptop in front of you. It should be logged in, with the file \vb{starting.py} open. If you don't think this is the case, then ask your session leader for help. Follow the steps below to change \vb{starting.py} to make the robot greet you.

\notebox{Any line starting with \code{\#} is a comment which the robot ignores. For example, \code{# This is a comment}. All other lines tell the robot to do something.}

\begin{enumerate}
    \item Find the line in the program which makes the robot say ``Hello World!''.
    \item Change the program to say``Hello [your names]!" (for example, ``Hello Jeremy!''). 
\end{enumerate}

\tipbox{\raggedright You will need to change the line which says \code{ev3.speaker.say("Hello World!")}}

Great, you've written your first program! Now, we need to send the program to the robot so it can run it.

\subsection*{Downloading a program to the robot}
The first step is to make sure the robot is plugged in to the laptop via the USB cable. If you aren't sure where to plug the robot in, or if you don't have a cable, ask your session leader for help.

On your screen, you should see the \vb{EV3DEV DEVICE BROWSER} at the bottom of the left panel. You may have to expand it by clicking on the small arrow next to it. If you don't see this, ask your session leader for help.

\warningbox{The next steps are very important! Make sure you follow them carefully. If you get stuck, ask your session leader for help.}

\newpage
Look for the small circle which is either green, yellow, or red. If it is green, like in picture 1 below, the robot is connected and ready to receive code. If it is yellow like in picture 2, the robot is currently in the process of connecting. If it is red, like in picture 3, the robot is not connected. If you are not connected, right click the robot's name, and then click \vb{Reconnect}.

\begin{figure}[h]
    \centering
    \includegraphics[width=0.45\textwidth]{images/green_light.png}
    \includegraphics[width=0.45\textwidth]{images/orange_light.png}
\end{figure}
\begin{figure}[h]
    \centering
        \includegraphics[width=0.45\textwidth]{images/red_light.png}
\end{figure}

The circle must be green like in picture 1 before you can download your program to the robot. If you can't make it turn green, ask your session leader for help.

\begin{enumerate}
    \item Hover the cursor to the right of where it says \vb{EV3DEV DEVICE BROWSER} and click on the right hand button which looks like an arrow pointing downwards.
    \begin{figure}[!h]
        \centering
        \includegraphics[width=0.32\textwidth]{images/down_arrow.png}
    \end{figure}
    \item Watch in the bottom right hand side of the laptop screen whilst the files and programs are downloaded to the robot. You will see a blue loading bar fill from left to right, and a message saying \vb{Download Complete} when it has finished.
    \begin{figure}[!h]
        \centering
        \includegraphics[width=0.52\textwidth]{images/download_example.png}
    \end{figure}
    \item Now, unplug the robot, and run the program again by following the steps in the \vb{Running a Program} section above. Listen carefully to make sure the robot says your name!
\end{enumerate}

\notebox{Now that you have unplugged the robot, the little circle will be red when you plug it back in. This is normal, it just means the robot is not connected to the laptop. You will have to reconnect the robot to the laptop by right clicking on the robot's name and selecting \vb{Reconnect}.}
