\section*{Getting the robot to move}
Now we want to write programs to get the robot to move in different ways. On the laptop, open the file \vb{moving.py} from the left panel. Your first tasks are to edit \vb{moving.py} to make the robot complete the following movements. Read below for the commands you need to make the robot move.

\begin{enumerate}
    \item Drive forwards 500 mm 
    \item Drive backwards 500 mm 
    \item Turn 180° clockwise
    \item Turn 90° anti-clockwise
    \item Drive in a square
\end{enumerate}

To drive the robot in a straight line, we use:
\begin{codeblock}
robot.straight(distance)
\end{codeblock}%
We must replace the \emph{word} \code{distance} with a \emph{number}. Here, it is the distance in millimeters that we want to move. 

To turn the robot on the spot, we use:
\begin{codeblock}
robot.turn(angle)
\end{codeblock}
Here, you replace \code{angle} with the number of degrees you want the robot to turn.

\tipbox{Hint: Values can be negative.\newline Bonus hint: You can list commands one after another.}

\thinkbox{Challenge: Can you write a program to make the robot drive along the racetrack?}