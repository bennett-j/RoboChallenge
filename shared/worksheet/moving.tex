\section*{Getting the robot to move}
Open the file \vb{moving.py}. You will see we have added the line \code{robot = DriveBase(...)}. It would be inconvenient if we had to control the robot through the motor speeds individually so instead we use \code{DriveBase} which provides some useful functions to control the robot:

\begin{itemize}
\item To drive the robot in a straight line for the given \code{distance} in mm, use:
\begin{codeblock}
robot.straight(distance)
\end{codeblock}

\item To turn the robot on the spot the given \code{angle} in degrees ($^\circ$), use:
\begin{codeblock}
robot.turn(angle)
\end{codeblock}

\item To combine driving forwards and turning, use this function. The robot will drive forwards at the speed \code{drive\_speed} in mm/s and turn at the speed \code{turn\_rate} in deg/s. This function starts the robot moving and it will continue until another command is given.
\begin{codeblock}
robot.drive(drive_speed, turn_rate)
\end{codeblock}
\end{itemize}

Modify \vb{moving.py} by to attempt the following movements. You need to select the appropriate function from above (\code{straight}, \code{turn} or \code{drive}) and replace the value (\code{distance}, \code{angle}, \code{drive_speed} and \code{turn_rate}) with the right number.

\begin{enumerate}
    \item Drive forwards 500 mm 
    \item Drive backwards 500 mm 
    \item Turn 180$^\circ$ on the spot both ways 
    \item Drive in a square 
    \item Drive in a circle 
\end{enumerate}

\tipbox{Hint: values can be negative.}
\thinkbox{Challenge: get the robot to move by controlling the motors without \code{DriveBase}.}