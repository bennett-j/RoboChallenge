\section*{Using the Reflectance Sensor - Important!}
You may have noticed that it was difficult and time consuming to complete the race track using manual commands. Robots use sensors to help make their own decisions by seeing the world around them.

\thinkbox{Imagine trying to walk in a square with your eyes closed. That would be difficult! It would be much easier if you could see. By using the sensor, we are allowing the robot to ``see".}

The robots you are using have \textbf{reflectance sensors}. You should see it sticking out of the front of the robot, shining a red light onto the table. If you can't see this, ask your session leader for help.

The sensor shines a light onto the floor underneath it, and measures how much is reflected back. If the floor is a light colour (for example, white) it will have a \textbf{high} reflectance value but if it's dark (for example black) the reflectance will be \textbf{low}.

\subsection*{Making sure the sensor is working}
% make all numbers sequential ?
\begin{enumerate}
    \item Put the robot on the piece of paper on your desk, with the wheels on the startline.
    \item On the robot, run the \vb{sensing.py} program. It should stop when it reaches the black finish line.
\end{enumerate}

% introduce exploring the sensor values and how it works
Your task now is to make the robot stop at the grey line in the middle of the paper. Let's measure the reflectance of the grey line.

% \subsection*{Measuring light reflectance}

\begin{enumerate}
    \setcounter{enumi}{2} 
    \item On the robot, run \vb{threshold\_measurer.py}. You will see lots of numbers on the robot's screen - these are the \textit{reflectance values} from the sensor.
    \item Move the robot around (with your hand) and see how the values change when the sensor is above different coloured surfaces.
    \item To stop the program, press the back button (top left button) on the robot.
    \item On the laptop, open \vb{sensing.py}.
\end{enumerate}

\warningbox{Before moving on: examine \vb{sensing.py} and figure out what made the robot stop on the black line? How can you make the robot stop at the grey line?}

% \newpage
% \subsection*{Stopping on the grey line}

\begin{enumerate}
    \setcounter{enumi}{6} 
    \item Find the lines in the program which define variables \code{measured_reflectance} and \code{stopping_threshold}. Read the \code{# NOTES FOR STUDENTS #} which discuss the variables and their purpose.
    \item Change the program's measured reflectance to make the robot stop at the grey line.
    \item Test your new program by sending it to the robot and seeing if your robot stops on the grey line - or if it still keeps going all the way to the black finish line.
\end{enumerate}

Did you manage to make the robot stop on the grey line? If you did, congratulations! If not, try again. Why not ask some of your class mates for help?

% \vspace{1em}
% \tipbox{Teamwork is the dreamwork!}